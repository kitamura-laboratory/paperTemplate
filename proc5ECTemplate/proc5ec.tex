%#!pdfpLaTeX
%
% 北村研究室用卒論予稿のTeXテンプレートファイル
% 本ファイルは非公式であり,下記で公開されているワードのテンプレートが公式である.
% https://www.kagawa-nct.ac.jp/EE/local/index.html (学内限定アクセス)
%
% 2020年1月18日 北村大地作成
%

\documentclass[a4j]{jsarticle}
\usepackage{nitkagawaproc5ec} % 予稿テンプレートクラスファイル
\usepackage{amsmath,amssymb} % 数式環境
\usepackage{bm} % 数式の太字斜体
\usepackage[dvipdfmx]{color}
\usepackage[dvipdfmx]{graphicx}
\usepackage{flushend} % 2段組みの最終ページの高さを揃える

\renewcommand{\baselinestretch}{0.8} % 行間設定(標準は0.8)

%%%%%%%%%%%%%%%%%%%%%%%%%%% 論文情報 %%%%%%%%%%%%%%%%%%%%%%%%%%%

%%%%% タイトル %%%%%
\title{日本語の研究タイトルをここに記述\\タイトルが2行になる場合はこの行にも記述可能}
\etitle{English title of research theme goes here}

%%%%% 著者 %%%%%
\author{高松 太郎}
\eauthor{Tarou Takamatsu}


%----- 図表題が英語の場合は次の2行を有効化 -----%
%\renewcommand{\figurename}{Fig.~} 
%\renewcommand{\tablename}{Table~}
%--------------------------------------------%

\begin{document}
\maketitle% タイトル生成

%----- ハイフン付きページ番号を表示する場合は次3行を有効化 -----%
%\setcounter{page}{1} % 開始ページ番号(3にすると3ページと4ページの2枚になる)
%\pagestyle{hyphenpage}
%\thispagestyle{hyphenpage}
%---------------------------------------------------------%

%----- ハイフン無しページ番号を表示する場合は次3行を有効化 -----%
%\setcounter{page}{1} % 開始ページ番号(3にすると3ページと4ページの2枚になる)
%\thispagestyle{plain}
%\pagestyle{plain}
%---------------------------------------------------------%

%----- ページ番号を削除する場合は次2行を有効化 -----%
\thispagestyle{empty}
\pagestyle{empty}
%-----------------------------------------------%

%%%%%%%%%%%%%%%%%%%%%%%%%%% 本文 %%%%%%%%%%%%%%%%%%%%%%%%%%%
%----------------------------------------------
\section{はじめに}
%----------------------------------------------

このファイルは卒業論文の予稿のワードテンプレートである.
このテンプレートに記載されている注意事項を熟読したうえ,予稿の作成に取り掛かること.

%----------------------------------------------
\section{本文の記述方法}
%----------------------------------------------

%----------------------------------------------
\subsection{フォーマット}
%----------------------------------------------

フォーマットはA4縦用紙に2段組み,2ページで作成する.
余白は上下が20~mm,左右が25~mmである.

%----------------------------------------------
\subsection{フォントと各種記号}
%----------------------------------------------

「1. はじめに」等の章タイトルは11~ptのゴシック系フォント(あるいはそれに準ずるフォント)で記述し,「2.1 フォーマットとフォント」等の節タイトルは10 ptのゴシック系フォントとする.
本文は全て10~ptの明朝系フォントとし,本文中の半角英数文字は明朝フォントまたはTimes系フォントとする(本テンプレートではTimes系に設定している).
なお,全角の英数文字を使うことが無いように注意すること.

段落の最初は必ず一字下げを挿入する.句読点は「、」と「。」の組み合わせか「,」と「.」の組み合わせに統一する.
括弧の記号については,(このように)全角の括弧を用いると適切なスペースが付与されるため,(このような)半角の括弧よりも見栄えが良くなる.
但し,英文中の括弧では必ず半角括弧とスペースを組み合わせて記述すること.
物理単位については,4.2~mや5.08~kgのように,半角スペースを空けたうえで付与することが望ましい.

%----------------------------------------------
\section{図表と数式の記述方法}
%----------------------------------------------

%----------------------------------------------
\subsection{図表の挿入方法}
%----------------------------------------------

図表は必ず本文上部または下部に挿入する.
本文の途中に挿入することの内容に注意せよ.
図の挿入例を図\ref{fig:sample}に示す.
このとき,図番号と図タイトルは図の中央下部に配置する.
また,印刷時の品質を向上させるために,図は可能な限りラスタ形式ではなくベクタ形式で作成し挿入することを推奨する.
さらに,表の挿入例を表\ref{table:sample}に示す.
表の場合は表番号と表タイトルを表の中央上部に配置する.
図番号・表番号及び図タイトル・表タイトルは本文と同じく10~ptの明朝系フォントとする.
なお,読みやすさの向上ために,図表の上下には1行分の余白を設けることが望ましい(図\ref{fig:sample}と表\ref{table:sample}の間や上下を参照のこと).

%-%-%-%-%-%-%-%-%
\begin{figure}[t]
  \centering
  \vspace{0pt} % 図上部の余白調整
  \includegraphics[width=0.8\columnwidth]{image.pdf}
  \vspace{0pt} % 図とキャプション間の余白調整
  \caption{図の挿入例}
  \vspace{0pt} % キャプション下部の余白調整
  \label{fig:sample}
\end{figure}
%-%-%-%-%-%-%-%-%

%-%-%-%-%-%-%-%-%
\begin{table}[t]
  \centering
  \caption{表の挿入例}
  \label{table:sample}
  \vspace{0pt} % キャプション下部の余白調整
  \begin{tabular}{|c|c|c|c|} \hline
    項目 & 数値1 & 数値2 & 数値3 \\ \hline
    A & 1 & 5 &  9 \\ \hline
    B & 2 & 6 & 10 \\ \hline
    C & 3 & 7 & 11 \\ \hline
    D & 4 & 8 & 12 \\ \hline
  \end{tabular}
  \vspace{10pt} % 表下部の余白調整
\end{table}
%-%-%-%-%-%-%-%-%

%----------------------------------------------
\subsection{数式の挿入方法}
%----------------------------------------------

本文中に数式を挿入する場合は,次に示すように中央寄せすると良い.
\begin{align}
  \nonumber  a^2 + b^2 = c^2
\end{align}
複数行の数式を挿入する場合は,次のように等号記号で左右位置をそろえる.
\begin{align}
  \nonumber  a^2 + b^2 &= c^2 \\
  \nonumber d^2 &= e^2 + f^2 \\
  \nonumber A + B + C &= E + F + G
\end{align}
横に長すぎる数式は改行を設けることで対処できる.
但し,可能な限り左辺と右辺の左右位置は重ならないようにすることが望ましい.
\begin{align}
  \nonumber  A &= B + C + D + E + F + G + H + I + J + K \\
  \nonumber &\phantom= + L + M + N + O + P
\end{align}
縦に幅を取る数式を挿入すると次のようになる.
\begin{align}
  \nonumber  f(x) = a_0 + \sum_{n=1}^{\infty} \left( a_n \cos \frac{ n\pi x }{ L } + b_n\sin \frac{ n\pi x }{ L } \right)
\end{align}
なお,数式中に登場した変数等は全て登場直後に説明されなければならない.
その一例を次式に示す.
\begin{align}
  \nonumber  x = \frac{ -b \pm \sqrt{ b^2 -4ac } }{ 2a }
\end{align}
ここで,$a$は2乗項の係数,$b$は1乗項の係数,$c$は定数項をそれぞれ示す.
本文全体で未定義の変数が残っていないか十分注意すること.

一度登場した数式を後の本文で引用する場合は,次式のように数式の右側に式番号を付与する.
\begin{align}
  A = \pi r^2 \label{eq:radius}
\end{align}
式番号は括弧書きの数字とし,登場する順番に番号を付与する.
式番号を用いて数式を引用する場合は「式(\ref{eq:radius})」又は「(\ref{eq:radius})式」と記述する.
どちらを用いても良いが,表記は本文全体で統一されている必要がある.

文中(インライン)に数式を挿入することもできる.
例えば,$f(x) = 1/2 \sin(\alpha + \beta)$のように記述できる.
ただし,縦に幅を取る数式はインラインで記述することは避けるべきである.
また分数も,インラインで表記する場合は$V = 4/3\pi r^3$のようにスラッシュ記号を用いて縦の幅を節約する工夫をする.

%----------------------------------------------
\subsection{参考文献の引用の挿入方法}
%----------------------------------------------

本文の適切な箇所で参考文献を引用する場合は,このように記述する\cite{sample1}.
この引用番号を表す記号は,予稿の末尾に記載されている参考文献リストと対応している\cite{sample2}.
但し,登場する順に番号を付与する.
なお,参考文献として不適切な文書を引用してはいけない.
例えば,信憑性が疑われるインターネット上の記事(ブログ,ウィキペディア,ニュース等)や学術的でない雑誌等は引用を避けるべきである.
厳正な査読を経て掲載された原著論文及び国際会議論文が最も適切であるが,確立された技術を網羅した教科書等を引用する場合もある.
その他,他者の学位論文(修士論文,博士論文等)を引用することもあるが,これらの多くは信憑性が疑われるものも多々あるので十分注意する.

%----------------------------------------------
\section{まとめ}
%----------------------------------------------

以上の方法で作成すると,正しい体裁の技術文書を作成することができる.
より読みやすく正しい技術文書を作成するためには,詳細を解説した参考書等を参考にすること.
また,予稿完成後も複数人で念入りにチェックを繰り返し,誤字や脱字,不適切な記述,数式の誤り等が無いように十分注意すること.

%----------------------------------------------
\section{注意事項}
%----------------------------------------------

本テンプレートは元々Wordでのみしか配布されていなかった卒業研究発表会予稿のテンプレートを,北村大地がLaTeXによる組版で可能な限り忠実に再現したファイルになります.
マージンや余白などには細心の注意を払いましたが,ボールドフォントの違い等の細かい点は組版ソフトの違いが存在する以上生じてしまいます.
あくまでも非公式のテンプレートとして自己責任で使用してください.

TeXのコンパイラは恐らく最もスタンダードであるpdfpLaTeXを使用することを想定しています.
他のコンパイラでの動作確認はしておりません.
また,今後の対応の予定もありません.
もう北村は疲れました.


%%%%% 参考文献 %%%%%
\begin{thebibliography}{9}% 10以上の文献数であれば99とする
%参考文献のフォントサイズを小さくしたい場合は下記2行のコメントアウトを解除
%\itemsep 3pt % 項目の間隔微調整用
%\fontsize{8pt}{10pt}\selectfont  % 項目のフォントサイズ微調整用

\bibitem{sample1}
  高松太郎,詫間花子,香川一郎,
  ``1つめの参考文献のタイトル,''
  {\em 2020年○○学会講演論文集}, pp.~278--280, 2020.

\bibitem{sample2}
  T.~Takamatsu, H. Takuma, and I. Kagawa, 
  ``Title of reference article,''
  {\em Proc. International Conference on XXX}, pp.~278--280, 2020.


\end{thebibliography}

\end{document}