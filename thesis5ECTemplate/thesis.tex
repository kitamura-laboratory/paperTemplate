%#!pdfpLaTeX
%
% 北村研究室用卒業論文・特別論文のTeXテンプレートファイル
% 本ファイルは非公式であり,表紙とアブストに関しては下記で公開されているワードの
% テンプレートを利用して作成したものが公式であるので,表紙とアブストはPDFにして
% 差し替えること.
% https://www.kagawa-nct.ac.jp/EE/local/index.html (学内限定アクセス)
%
% 2020年1月17日 北村大地作成
%

%%%%%%%%%%%%%%%%%%%%%%%%%%% 論文情報 %%%%%%%%%%%%%%%%%%%%%%%%%%%
%%%%% テンプレート選択 %%%%%
\documentclass[honka]{nitkcthesis}%卒論(本科5年)日本語用
%\documentclass[honka,english]{nitkcthesis}%卒論(本科5年)英語用
%\documentclass[senkouka]{nitkcthesis}%特論(専攻科2年)日本語用
%\documentclass[senkouka,english]{nitkcthesis}%特論(専攻科2年)英語用

%%%%% タイトル %%%%%
\title{\underline{北村研究室用}\\\underline{卒業論文テンプレートファイル}}
%\titlewidth{}% タイトル幅 (指定するときは単位つきで)

%%%%% 著者 %%%%%
\author{高専 太郎}
\eauthor{Taro Kosen}% Copyright表示で使われる

%%%%% 指導教員名 %%%%%
\supervisor{北村 大地 助教}% 1つ引数をとる (役職まで含めて書く)

%%%%% 提出年月 %%%%%
\date{令和2年 2月 4日}

%%%%% \usepackage等のプリアンブル宣言(macros.texに記載) %%%%%
\usepackage{bm}
\usepackage{amsmath, amssymb}
\usepackage[dvipdfmx]{color}
\usepackage[dvipdfmx]{graphicx}
\usepackage{tabularx}
\usepackage{booktabs}
\usepackage{multirow}
\usepackage{setspace}
\usepackage{amsthm}
\usepackage[caption=false]{subfig}
\usepackage[numbers,sort]{natbib}

\theoremstyle{definition}
\newtheorem{theo}{定理}[chapter]
\newtheorem{defi}{定義}[chapter]
\newtheorem{lemm}{補題}[chapter]
\renewcommand{\proofname}{\textbf{証明}}

%% definition
\newcommand{\J}{\mathrm{j}}
\newcommand{\diag}{\mathop{\mathrm{diag}}}

\newcommand{\mtr}[1]{#1^{\mathsf{T}}}
\newcommand{\ctr}[1]{#1^{\mathsf{H}}}
\newcommand{\inv}[1]{#1^{-1}}
\newcommand{\cinv}[1]{#1^{-\mathsf{H}}}
\newcommand{\tinv}[1]{#1^{-\mathsf{T}}}
\newcommand{\conj}[1]{#1^*}

\newcommand{\tbm}[1]{\tilde{\bm{#1}}}
\newcommand{\tsf}[1]{\tilde{\mathsf{#1}}}

\newcommand{\vw}{\bm{w}}
\newcommand{\mW}{\bm{W}}
\newcommand{\vwhat}{\widehat{\bm{w}}}
\newcommand{\mWhat}{\widehat{\bm{W}}}

\newcommand{\hhat}{\widehat{h}}
\newcommand{\rhat}{\widehat{r}}

\newcommand{\argmax}{\mathop{\mathrm{arg~max}}\limits}
\newcommand{\argmin}{\mathop{\mathrm{arg~min}}\limits}

\renewcommand{\Re}{\mathop{\mathrm{Re}}}
\renewcommand{\Im}{\mathop{\mathrm{Im}}}

\newcommand{\unit}[1]{~\mathrm{#1}}
\newcommand{\Unit}[1]{~\mathrm{\left[#1\right]}}

\renewcommand{\qedsymbol}{$\blacksquare$}

\bibliographystyle{IEEEtran}

\makeatletter
\def\bstctlcite{\@ifnextchar[{\@bstctlcite}{\@bstctlcite[@auxout]}}
\def\@bstctlcite[#1]#2{\@bsphack
\@for\@citeb:=#2\do{%
\edef\@citeb{\expandafter\@firstofone\@citeb}%
\if@filesw\immediate\write\csname #1\endcsname{\string\citation{\@citeb}}\fi}%
\@esphack}
\makeatother


\begin{document}
\bstctlcite{IEEEexample:BSTcontrol} % BibTeXのIEEEtranで同一著者の横線表示を防止

\maketitle% タイトル生成

%%%%%%%%%%%%%%%%%%%%%%%%%%% 前文 %%%%%%%%%%%%%%%%%%%%%%%%%%%
\frontmatter

%%%%% English title %%%%%
\etitle{Template File of Graduate Thesis Produced Only for Kitamura Laboratory}

%%%%% Abstract %%%%%
\eabstract{
English abstract goes here. English abstract goes here. English abstract goes here. English abstract goes here. 
English abstract goes here. English abstract goes here. English abstract goes here. English abstract goes here. 
English abstract goes here. English abstract goes here. English abstract goes here. English abstract goes here. 
English abstract goes here. English abstract goes here. English abstract goes here. English abstract goes here. 
English abstract goes here. English abstract goes here. English abstract goes here. English abstract goes here. 
English abstract goes here. English abstract goes here. English abstract goes here. English abstract goes here. 
English abstract goes here. English abstract goes here. English abstract goes here. English abstract goes here. 
English abstract goes here. English abstract goes here. English abstract goes here. English abstract goes here.  
}

%%%%% 概要 %%%%%
\abstract{
日本語の概要をここに記述.日本語の概要をここに記述.日本語の概要をここに記述.日本語の概要をここに記述.
日本語の概要をここに記述.日本語の概要をここに記述.日本語の概要をここに記述.日本語の概要をここに記述.
日本語の概要をここに記述.日本語の概要をここに記述.日本語の概要をここに記述.日本語の概要をここに記述.
日本語の概要をここに記述.日本語の概要をここに記述.日本語の概要をここに記述.日本語の概要をここに記述.
日本語の概要をここに記述.日本語の概要をここに記述.日本語の概要をここに記述.日本語の概要をここに記述.
日本語の概要をここに記述.日本語の概要をここに記述.日本語の概要をここに記述.日本語の概要をここに記述.
日本語の概要をここに記述.日本語の概要をここに記述.日本語の概要をここに記述.日本語の概要をここに記述.
日本語の概要をここに記述.日本語の概要をここに記述.日本語の概要をここに記述.日本語の概要をここに記述.
日本語の概要をここに記述.日本語の概要をここに記述.日本語の概要をここに記述.日本語の概要をここに記述.
日本語の概要をここに記述.日本語の概要をここに記述.日本語の概要をここに記述.日本語の概要をここに記述.
日本語の概要をここに記述.日本語の概要をここに記述.日本語の概要をここに記述.日本語の概要をここに記述.
日本語の概要をここに記述.日本語の概要をここに記述.日本語の概要をここに記述.日本語の概要をここに記述.
日本語の概要をここに記述.日本語の概要をここに記述.日本語の概要をここに記述.日本語の概要をここに記述
}

\keywords{Keyword1, Keyword2, Keyword3}

\makeseparatedabstract
%\makeabstract
%%%%% 目次 %%%%%
%\tableofcontents % ページ番号を削除しない目次
%----- ページ番号を削除した目次 -----%
{\makeatletter
\let\ps@jpl@in\ps@empty
\makeatother
\pagestyle{empty}
\tableofcontents
\clearpage}
%---------------------------------%

%%%%%%%%%%%%%%%%%%%%%%%%%%% 本文 %%%%%%%%%%%%%%%%%%%%%%%%%%%
\mainmatter

%%%%% 第1章 %%%%%
\chapter{緒言}
\label{chap:intro}

%----------------------------------------------
\section{本論文の背景}
%----------------------------------------------
本論文の背景をここに記載.本論文の背景をここに記載.本論文の背景をここに記載.本論文の背景をここに記載.
本論文の背景をここに記載.本論文の背景をここに記載.本論文の背景をここに記載.本論文の背景をここに記載.
本論文の背景をここに記載.本論文の背景をここに記載.本論文の背景をここに記載.本論文の背景をここに記載.
本論文の背景をここに記載.本論文の背景をここに記載.本論文の背景をここに記載.本論文の背景をここに記載.
本論文の背景をここに記載.本論文の背景をここに記載.本論文の背景をここに記載.本論文の背景をここに記載.
本論文の背景をここに記載.本論文の背景をここに記載.本論文の背景をここに記載.本論文の背景をここに記載.
本論文の背景をここに記載.本論文の背景をここに記載.本論文の背景をここに記載.本論文の背景をここに記載.
本論文の背景をここに記載.本論文の背景をここに記載.本論文の背景をここに記載.本論文の背景をここに記載.
本論文の背景をここに記載.本論文の背景をここに記載.本論文の背景をここに記載.本論文の背景をここに記載.
本論文の背景をここに記載.本論文の背景をここに記載.本論文の背景をここに記載.本論文の背景をここに記載.
本論文の背景をここに記載.本論文の背景をここに記載.本論文の背景をここに記載.本論文の背景をここに記載.
本論文の背景をここに記載.本論文の背景をここに記載.本論文の背景をここに記載.本論文の背景をここに記載.
本論文の背景をここに記載.本論文の背景をここに記載.本論文の背景をここに記載.本論文の背景をここに記載.
本論文の背景をここに記載.本論文の背景をここに記載.本論文の背景をここに記載.本論文の背景をここに記載.
本論文の背景をここに記載.本論文の背景をここに記載.本論文の背景をここに記載.本論文の背景をここに記載.
本論文の背景をここに記載.本論文の背景をここに記載.本論文の背景をここに記載.本論文の背景をここに記載.
本論文の背景をここに記載.本論文の背景をここに記載.本論文の背景をここに記載.本論文の背景をここに記載.
本論文の背景をここに記載.本論文の背景をここに記載.本論文の背景をここに記載.本論文の背景をここに記載.
本論文の背景をここに記載.本論文の背景をここに記載.本論文の背景をここに記載.本論文の背景をここに記載.
本論文の背景をここに記載.本論文の背景をここに記載.本論文の背景をここに記載.本論文の背景をここに記載.
本論文の背景をここに記載.本論文の背景をここに記載.本論文の背景をここに記載.本論文の背景をここに記載.
本論文の背景をここに記載.本論文の背景をここに記載.本論文の背景をここに記載.本論文の背景をここに記載.
本論文の背景をここに記載.本論文の背景をここに記載.本論文の背景をここに記載.本論文の背景をここに記載.
本論文の背景をここに記載.本論文の背景をここに記載.本論文の背景をここに記載.本論文の背景をここに記載.
本論文の背景をここに記載.本論文の背景をここに記載.本論文の背景をここに記載.本論文の背景をここに記載.
本論文の背景をここに記載.本論文の背景をここに記載.本論文の背景をここに記載.本論文の背景をここに記載.
本論文の背景をここに記載.本論文の背景をここに記載.本論文の背景をここに記載.本論文の背景をここに記載.
本論文の背景をここに記載.本論文の背景をここに記載.本論文の背景をここに記載.本論文の背景をここに記載.
本論文の背景をここに記載.本論文の背景をここに記載.本論文の背景をここに記載.本論文の背景をここに記載.
本論文の背景をここに記載.本論文の背景をここに記載.本論文の背景をここに記載.本論文の背景をここに記載.

%-%-%-%-%-%-%-%-%
\begin{figure}[!t]
\centering
\includegraphics[width=0.95\hsize]{ch_intro/sample.pdf}
\caption{Sample figure.}
\label{fig:intro:sample}
\end{figure}
%-%-%-%-%-%-%-%-%

%----------------------------------------------
\section{本論文の目的}
%----------------------------------------------
本論文の目的をここに記載.本論文の目的をここに記載.本論文の目的をここに記載.本論文の目的をここに記載.
本論文の目的をここに記載.本論文の目的をここに記載.本論文の目的をここに記載.本論文の目的をここに記載.
本論文の目的をここに記載.本論文の目的をここに記載.本論文の目的をここに記載.本論文の目的をここに記載.
本論文の目的をここに記載.本論文の目的をここに記載.本論文の目的をここに記載.本論文の目的をここに記載.
本論文の目的をここに記載.本論文の目的をここに記載.本論文の目的をここに記載.本論文の目的をここに記載.
本論文の目的をここに記載.本論文の目的をここに記載.本論文の目的をここに記載.本論文の目的をここに記載.
本論文の目的をここに記載.本論文の目的をここに記載.本論文の目的をここに記載.本論文の目的をここに記載.
本論文の目的をここに記載.本論文の目的をここに記載.本論文の目的をここに記載.本論文の目的をここに記載.
本論文の目的をここに記載.本論文の目的をここに記載.本論文の目的をここに記載.本論文の目的をここに記載.
本論文の目的をここに記載.本論文の目的をここに記載.本論文の目的をここに記載.本論文の目的をここに記載.
本論文の目的をここに記載.本論文の目的をここに記載.本論文の目的をここに記載.本論文の目的をここに記載.
本論文の目的をここに記載.本論文の目的をここに記載.本論文の目的をここに記載.本論文の目的をここに記載.
本論文の目的をここに記載.本論文の目的をここに記載.本論文の目的をここに記載.本論文の目的をここに記載.
本論文の目的をここに記載.本論文の目的をここに記載.本論文の目的をここに記載.本論文の目的をここに記載.
本論文の目的をここに記載.本論文の目的をここに記載.本論文の目的をここに記載.本論文の目的をここに記載.
本論文の目的をここに記載.本論文の目的をここに記載.本論文の目的をここに記載.本論文の目的をここに記載.
本論文の目的をここに記載.本論文の目的をここに記載.本論文の目的をここに記載.本論文の目的をここに記載.
本論文の目的をここに記載.本論文の目的をここに記載.本論文の目的をここに記載.本論文の目的をここに記載.
本論文の目的をここに記載.本論文の目的をここに記載.本論文の目的をここに記載.本論文の目的をここに記載.
本論文の目的をここに記載.本論文の目的をここに記載.本論文の目的をここに記載.本論文の目的をここに記載.
本論文の目的をここに記載.本論文の目的をここに記載.本論文の目的をここに記載.本論文の目的をここに記載.
本論文の目的をここに記載.本論文の目的をここに記載.本論文の目的をここに記載.本論文の目的をここに記載.
本論文の目的をここに記載.本論文の目的をここに記載.本論文の目的をここに記載.本論文の目的をここに記載.
本論文の目的をここに記載.本論文の目的をここに記載.本論文の目的をここに記載.本論文の目的をここに記載.
本論文の目的をここに記載.本論文の目的をここに記載.本論文の目的をここに記載.本論文の目的をここに記載.
本論文の目的をここに記載.本論文の目的をここに記載.本論文の目的をここに記載.本論文の目的をここに記載.
本論文の目的をここに記載.本論文の目的をここに記載.本論文の目的をここに記載.本論文の目的をここに記載.
本論文の目的をここに記載.本論文の目的をここに記載.本論文の目的をここに記載.本論文の目的をここに記載.
本論文の目的をここに記載.本論文の目的をここに記載.本論文の目的をここに記載.本論文の目的をここに記載.
本論文の目的をここに記載.本論文の目的をここに記載.本論文の目的をここに記載.本論文の目的をここに記載.
本論文の目的をここに記載.本論文の目的をここに記載.本論文の目的をここに記載.本論文の目的をここに記載.
本論文の目的をここに記載.本論文の目的をここに記載.本論文の目的をここに記載.本論文の目的をここに記載.
本論文の目的をここに記載.本論文の目的をここに記載.本論文の目的をここに記載.本論文の目的をここに記載.
本論文の目的をここに記載.本論文の目的をここに記載.本論文の目的をここに記載.本論文の目的をここに記載.
本論文の目的をここに記載.本論文の目的をここに記載.本論文の目的をここに記載.本論文の目的をここに記載.
本論文の目的をここに記載.本論文の目的をここに記載.本論文の目的をここに記載.本論文の目的をここに記載.
本論文の目的をここに記載.本論文の目的をここに記載.本論文の目的をここに記載.本論文の目的をここに記載.
本論文の目的をここに記載.本論文の目的をここに記載.本論文の目的をここに記載.本論文の目的をここに記載.
本論文の目的をここに記載.本論文の目的をここに記載.本論文の目的をここに記載.本論文の目的をここに記載.
本論文の目的をここに記載.本論文の目的をここに記載.本論文の目的をここに記載.本論文の目的をここに記載.

%----------------------------------------------
\section{本論文の構成}
%----------------------------------------------
\ref{chap:conv}章では,なんらかの手法\cite{Kitamura2016taslp}及び
関連の深い各種既存手法\cite{Kitamura2016IWAENC}について述べる.
\ref{chap:con}章では,本論文の結論を述べる.


%%%%% 第2章 %%%%%
\chapter{従来手法}
\label{chap:conv}

%----------------------------------------------
\section{まえがき}
%----------------------------------------------
本章では,何らか法の従来手法を説明する.
まず\ref{sec:conv:something}節では,何らかの分野で従来より用いられる何とかについて何とかを導入する.
\ref{sec:conv:somewhat}節では,なんとかについて述べる.

%----------------------------------------------
\section{何らか法}
\label{sec:conv:something}
%----------------------------------------------

何らか法とは,何らか何らか何とかかんとかであり,次式で表される
\begin{align}
  \hat{\bm{\theta}} = \argmin_{\bm{\theta} \in S} \ \mathcal{J}(\bm{\theta})
\end{align}

%----------------------------------------------
\section{何とか法}
\label{sec:conv:somewhat}
%----------------------------------------------

%----------------------------------------------
\subsection{表記の定義}
\label{sec:conv:somewhat:definition}
%----------------------------------------------
何とかかんとかと書ける.

%----------------------------------------------
\subsection{何とか法の導出}
\label{sec:conv:somewhat:derivation}
%----------------------------------------------
何とかかんとかと導ける.

%----------------------------------------------
\section{本章のまとめ}
%----------------------------------------------
本章では,何らか法の従来手法について説明した.
次章以降では,より詳細な何とかや何らか法の適用範囲の拡大を達成するために,
\ref{sec:conv:somewhat:derivation}節で導入した何とか法の発展的な理論拡張を提案する.


%%%%% 第3章 %%%%%
\chapter{結言}
\label{chap:con}

本論文では,何とか法に基づく何とかの精度を更に改善させるための理論的拡張について述べた.
特に,何とかかんとか.

\ref{chap:intro}章では,近年の何とか法に関する研究を紹介し,
本論文の目的について述べた.

\ref{chap:conv}章では,なんとかについて述べた.

最後に今後の課題を述べる.
現在の何とか法は初期値や局所解による性能のばらつきが大きく,
より安定に高い性能を実現する何とかアルゴリズムの検討は必須である.
特に,何とかかんとか.

%%%%%%%%%%%%%%%%%%%%%%%%%%% 後付 %%%%%%%%%%%%%%%%%%%%%%%%%%%
\backmatter

%%%%% 謝辞 %%%%%
\chapter{謝辞}

下記の謝辞はあくまでも記載例です.
まさか記載例の通りの文言で謝辞を述べるつもりではありませんよね?
下記を参考にするのは結構ですが,必ず自分の言葉で感謝の気持ちを伝えるべき人間に伝えましょう.

本論文は,香川高等専門学校電気情報工学科北村研究室にて行われた研究に基づくものです.

まず,本研究を進めるにあたり,ご多忙のところ熱心に
ご指導くださいました指導教員の北村大地講師に心より感謝申し上げます.
北村大地講師には,論文執筆や研究に関する議論など,細部にわたるまで
丁寧にご指導いただきました.

本論の副査である○○○○教授と○○○○准教授には,論文の構成や記述に関して
大変有益な助言を頂き,大変お世話になりました.
ここに厚く御礼申し上げます.

〇〇株式会社の○○○○博士には,共同研究ミーティングを通じ,数多くの有益な
ご指摘のほか,様々なご支援をいただきました.
心より感謝申し上げます.

○○大学の〇〇○○教授には,共同研究を通じ多数のご支援とご助言をいただきました.
心より感謝申し上げます.

北村研究室の先輩である専攻科〇年の○○○○氏には,
○○に関するアドバイス等をはじめ,数々のご支援をいただきました.
また,北村研究室同期の○○○○氏・○○○○氏・○○○○氏,
後輩の○○○○氏・○○○○氏には,ゼミや日頃のディスカッションのほか,
〇年に亘る研究室生活を様々な面で支えていただきました.
ここに感謝申し上げます.

最後になりますが,現在に至るまで私の学生生活を金銭的に支え,
暖かく見守って下さった両親には感謝の念に堪えません.
これまで本当にありがとうございました.


%%%%% 参考文献(直接書く場合) %%%%%
\begin{thebibliography}{99}
  \bibitem{Kitamura2016taslp}
    D.~Kitamura, N.~Ono, H.~Sawada, H.~Kameoka, and H.~Saruwatari, 
    ``Determined blind source separation unifying independent vector analysis and nonnegative matrix factorization,''
    \emph{IEEE/ACM Transactions on Audio, Speech, and Language Processing}, vol. 24, no. 9, pp. 1626--1641, 2016.

  \bibitem{Kitamura2016IWAENC}
    D.~Kitamura, N.~Ono, H.~Saruwatari, Y.~Takahashi, and K. Kondo, 
    ``Discriminative and reconstructive basis training for audio source separation with semi-supervised nonnegative matrix factorization,''
    \emph{in Proceedings of International Workshop on Acoustic Signal Enhancement}, 2016, pp. 1--5.
\end{thebibliography}

%%%%% 参考文献(BibTeXを使う場合) %%%%%
% \bibliography{ref_abb_full,references}

%%%%% 発表文献一覧 %%%%%
{
\chapter*{発表文献一覧}
\newcommand{\myname}[1]{\textbf{\underline{#1}}}

\section*{査読付き国際会議}
\begin{enumerate}
  \item \myname{Taro~Kosen} and Daichi~Kitamura,
    ``Awesome method for surprising something,'' 
    in Proceedings of \emph{{IEEE} International Conference on Something Awesome}, 
    2018, pp. 100--103.
  \item \myname{Taro~Kosen}, Hanako~Kagawa, and Daichi~Kitamura,
    ``Very awesome method for surprising something,'' 
    in Proceedings of \emph{{IEEE} International Conference on Something Awesome}, 
    2019, pp. 100--103.
\end{enumerate}


\section*{国内学会}
\begin{enumerate}
  \item  \myname{高専太郎}, 北村大地, 
    ``驚くべき何かの為の素晴らしい手法,''
    日本何らか学会 2018年春季研究発表会講演論文集, 1-1-10, pp. 100--101, 2018.
  \item  \myname{高専太郎}, 香川花子, 北村大地, 
    ``さらに驚くべき何かの為の素晴らしい手法,''
    日本何らか学会 2019年春季研究発表会講演論文集, 1-1-10, pp. 100--101, 2019.
\end{enumerate}


\section*{受賞}
\begin{enumerate}
  \item 日本何らか学会 第10回 優秀学生発表賞
\end{enumerate}
}


%%%%%%%%%%%%%%%%%%%%%%%%%%% 付録 %%%%%%%%%%%%%%%%%%%%%%%%%%%
\appendix

%%%%% 付録A %%%%%
\chapter{補助関数法で利用される不等式}
\label{chap:ineq}

%----------------------------------------------
\section{接線不等式}
%----------------------------------------------

\begin{lemm} \label{lem:aux:sessen} (接線不等式)
  $f(x)$が凹関数であるとき,以下の不等式が成立する.
  \begin{align}
    f(x) \leq f'(\bar{x}) (x - \bar{x}) + f(\bar{x})
  \end{align}
  不等式中の等号が成立するための条件は$x = \bar{x}$である.
\end{lemm}

%----------------------------------------------
\section{Jensenの不等式}
%----------------------------------------------

\begin{lemm} \label{lem:aux:jensen} (Jensenの不等式)
$\alpha_{i} > 0$を,$\sum_{i} \alpha_{i} = 1$を満たす補助変数とする.
関数$f(x)$が凸関数であるとき,$x_{i}\ (i = 1,\ldots,I)$に対して以下の不等式が成立する.
\begin{align}
  f \left( \sum_{i=1}^{I} \alpha_{i} x_{i} \right) \leq \sum_{i=1}^{I} \alpha_{i} f(x_{i})
\end{align}
$f(x)$が狭義凸関数であるとき,不等式中の等号が成立するための条件は
$x_1 = \cdots = x_i = \cdots = x_I$である.
\end{lemm}


\end{document}
